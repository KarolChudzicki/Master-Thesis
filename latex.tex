\subsection{Object locatization in robotics vision}

\subsection{Visual Servoing and Feedback Control}

\subsection{Prediction and Filtering Techniques for Robust Tracking}
A paper ... proposes a visual servo control alogorithm to estimate a positon of
a moving object. Instead of using position based control for the project which relies on
precise models of the camera and the robot, authors propose a image based control which
caluculate the error and postion of the object basing only on the information from the camera
plane. (maybe explain it more) This approach eliminates the errors from robot and camera models.
To further improve the reliability of their system, reaserchers used Kalman filter to
enhance pose estimation. The results demonstrated that the incorrect or inaccurate data
from a camera due to it's 

% Note
Try implementing image based control and position based control and determine which is better
maybe the image based control is enough?

% 28.06

To get an extimate of what the conveyor speed is.
Time measured with stopwatch and 100cm ruler
30.75s
30.75s
30.95s
30.81s
30.73s


















Goals for the trip to Austria within thesis:

- 3 side by side responses of PID and describe them
- x and y speed plots and describe them

- write about KFF
- about this zero problem

- write about cos and hermit smoothing of alpha
- write that Kalman filter would be great for this
- write about the y speed control
- add photos that you did!!

- go through the whole thesis and read it, maybe add some 1-2 papers to the analysis section

- when you finish this send it to the promotor !!!! set a deadline Wednesday evening!


!!!!!!!!!!!!!!!!!!!!!!!!!!!!!!!!!!!!!!!!!!!!!!!!!!!!!!!!!!!!!!!!!!!!!!!!!!!!!!!!!
New idea. Update predicted position whenever the camera records a new position!!!
!!!!!!!!!!!!!!!!!!!!!!!!!!!!!!!!!!!!!!!!!!!!!!!!!!!!!!!!!!!!!!!!!!!!!!!!!!!!!!!!!

